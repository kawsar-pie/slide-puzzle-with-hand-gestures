% ------------------------------------------------------------------------------
% Chapter 1
% Delete this content and replace with your own
% ------------------------------------------------------------------------------
\chapter{Introduction} % enter the name of the chapter here
Slide puzzle games have been around for decades and are a staple in the gaming world. They offer a challenging and engaging gameplay experience that requires players to use their problem-solving skills to rearrange puzzle pieces and form a complete picture. The latest evolution in slide puzzle games is the use of hand gestures to control the game, rather than traditional mouse and keyboard controls. This new development takes the classic gameplay experience to the next level, making it more immersive and interactive for players.

This new slide puzzle game is developed using Python, PyGame, and OpenCV. Python is a popular and versatile programming language that is used for a wide range of applications, including gaming. PyGame is a library for Python that is specifically designed for game development, providing a range of tools and functions that make it easier to create games. OpenCV, on the other hand, is a computer vision library that is used for real-time image processing and object recognition. Together, these technologies provide the foundation for this innovative slide puzzle game that uses hand gestures to control the game.

In the game, players are presented with a puzzle board containing several pieces that they must rearrange to form a complete picture. The hand gesture recognition system uses the camera on the player's device to detect and track number of fingers which are up, changing the number of fingers up the players can move the puzzle pieces around the board, either up-down or right-left, allowing players to interact with the game in a more natural and intuitive way. The game provides real-time feedback on the player's hand gestures, making it easy to see if the movement was successful or not.

